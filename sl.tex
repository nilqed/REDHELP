\begin{document}



\section{Preliminaries}

\subsection{Primitive Data Types}
\begin{Type}{integer}
Integer numbers:
Integers are also called "fixed" numbers. The magnitude of
an integer is unrestricted. Integers in the LISP input stream are
an arbitrary number of integer digits, eventually preceded by
a plus or minus sign.
\begin{Examples}
22\\
-31415926585\\
\end{Examples}
\end{Type} 

\begin{Type}{floating}
Floating point numbers: The precision of floating point
numbers is determined solely by the implementation. In BNF floating
point numbers are recognized by the grammar:
\begin{verbatim}
     <base> ::= <unsigned-integer>.|.<unsigned-integer>|
                <unsigned-integer>.<unsigned-integer>
                <unsigned-floating> ::= <base>|
                <base>E<unsigned-integer>|
                <base>E-<unsigned-integer>|
                <base>E+<unsigned-integer>
     <floating> ::= <unsigned-floating>|
                    +<unsigned-floating>|-<unsigned-floating>
\end{verbatim}
\begin{Examples}
3.1415\\
17.0\\
-22e100\\
1.1e-5
\end{Examples}
\end{Type}

\begin{Type}{id}
An identifier is a string of characters which may have the
following items associated with it.


     print name: The characters of the identifier.

     flags: An  identifier may  be  tagged with  a flag.    Access  is
          by  the \nameref{flag},  \nameref{remflag}and  
          \nameref{flagp} functions.

     properties: An  identifier  may  have  an   indicator-value  pair
          associated  with it.    Access  is  by the  
          \nameref{put}, \nameref{get}, and \nameref{remprop}
          functions.

     values: An identifier may have a value  associated with
          it.    Access to  values  is by  \nameref{set} \nameref{setq}
          The method by  which the  value
          is attached  to  the identifier  is  known as  the  binding
          type, being one of 
          \nameref{Local Binding}, \nameref{Global Binding},
          or \nameref{Fluid Binding}.

     functions:
          An identifier may have a function or macro associated  with
          it.    Access is  by the  
          \nameref{putd}, \nameref{getd}, and \nameref{remd} functions.
          An identifier  may not  have both  a function  and a  value
          associated with it.


     \name{oblist} entry: An  identifier may  be entered and  removed from  a
          structure called  the \nameref{oblist}.  Its presence  on the  \name{oblist}
          does not directly affect  the other properties.  Access  to
          the \name{oblist} is by the 
          \nameref{intern}, \nameref{remob}, and \nameref{read}
          functions.        

     The  maximum  length  of  a  Standard  LISP  identifier   is  24
     characters  (excluding occurrences  of the  escape character  !)
     but  an  implementation may  allow  more.    Special  characters
     (digits in the first position and punctuation) must  be prefixed
     with  an escape  character,  an !    in  Standard LISP.  In  BNF
     identifiers are recognized by the grammar:
\begin{verbatim}
     <special-character> ::= !<any-character>
     <alphabetic> ::=

          A|B|C|D|E|F|G|H|I|J|K|L|M|N|O|P|Q|R|S|T|U|V|W|X|Y|Z|
          a|b|c|d|e|f|g|h|i|j|k|l|m|n|o|p|q|r|s|t|u|v|w|x|y|z
     <lead-character> ::= <special-character>|<alphabetic>
     <regular-character> ::= <lead-character>|<digit>
     <last-part> ::= <regular-character> |
                     <last-part><regular-character>
     <id> ::= <lead-character>|<lead-character><last-part>

     Note:    Using  lower case  letters  in  identifiers  may  cause
     portability  problems.   Lower  case letters  are  automatically
     converted to upper case when the \nameref{!*RAISE} flag is T.
\end{verbatim}

\begin{Examples}
a\\
Hugo\\
!1!-otto\\
!*raise\\
this!-is!-a!-long!-id\\
!U!P!P!E!R!-and!-!l!o!w!e!r\\
\end{Examples}
\end{Type}


\begin{Type}{string}
A set of characters enclosed in double quotes as
in "THIS IS A STRING". A quote is included by doubling it as in "HE
SAID, ""LISP""". The maximum size of strings is 80 characters but an
implementation may allow more. Strings are not part of the \nameref{oblist} and
are considered constants like \nameref{number}s, \nameref{vector}s, 
and \nameref{function-pointer}s.
\end{Type}

\begin{Type}{dotted-pair}
\index{car}\index{cdr}
A dotted pair is a primitive structure which has a left and right part.
A notation called {\em dot-notation} is used for dotted pairs and
takes the form:
\begin{verbatim}
     (<left-part> .  <right-part>)
\end{verbatim}

The <left-part> is known as the \nameref{car} portion and the
<right-part> as the \nameref{cdr} portion. The left and right parts may be of any type.
Spaces are used to resolve ambiguity with floating point numbers.

When <left-part> or <right-part> are dotted-pairs themselves,
the \nameref{list-notation} is often more convenient.
\end{Type}

\begin{Type}{vector}
A vector is a primitive uniform structure in which
an integer index is used to access random values in the structure. The
individual elements of a vector may be of any type. Access to vectors
is restricted to functions \nameref{putv}, \nameref{getv}, and 
\nameref{upbv}.
A notation  for vectors, vector-notation, has  the
elements of a vector surrounded by square brackets
\begin{verbatim}                                                     
     <elements> ::= <any>|<any> <elements>
     <vector> ::= [<elements>]
\end{verbatim}
\begin{Examples}
[1 2 3 5 7 11 13 17 19 23]\\
[nil (a) (a . a)]\\
[[1 2 3 4 5][2 4 6 8 10][3 6 9 12 15]]\\
\end{Examples}
\end{Type}

\begin{Type}{function-pointer}

 An  implementation  may have  functions  which  deal
     with  specific data  types other  than those  listed.   The  use
     of  these entities  is to  be avoided  with the  exception of  a
     restricted  use of  the \name{function-pointer},  an  access method  to
     compiled EXPRs  and FEXPRs (see \nameref{Function Types}).  

     A particular 
     \name{function-pointer}  must
     remain  valid throughout execution.   Systems  which change  the
     location of a function must use either an  indirect reference or
     change all occurrences  of the associated value.  There are  two
     classes of  use of function-pointers, those which are  supported
     by Standard LISP  but are not well defined, and those which  are
     well defined.
\end{Type}

\subsection{Classes of Primitive Data Types}
\begin{Introduction}{Type Classes}
The classes of primitive types are a notational convenience for
describing the properties of functions.
\end{Introduction}

\begin{Type}{boolean}
The  set of  global variables  \{\nameref{T}, \nameref{NIL}\}, or  their  respective
     values, \{T, NIL\}.
\end{Type}

\begin{Type}{extra-boolean}
Any value in the system.  Anything that is not \nameref{NIL} has
     the boolean interpretation T.
\end{Type}

\begin{Type}{ftype}
 The class of definable  function types.  The set of ids \{EXPR,
     FEXPR, MACRO\}. See \nameref{Function Types}.
\end{Type}

\begin{Type}{number}
The set of \{\nameref{integer}, \nameref{floating}\}.
\end{Type}


\begin{Type}{constant}
The set  of \{\nameref{integer},  \nameref{floating}, \nameref{string},  
      \nameref{vector},  \nameref{function-pointer} \}.  
    Constants evaluate to themselves (see  \nameref{eval})
\end{Type}

\begin{Type}{any}
The set of  \{\nameref{integer}, \nameref{floating}, \nameref{string}, 
      \nameref{id}, \nameref{dotted-pair},  \nameref{vector},
     \nameref{function-pointer}\}.   An  S-expression is another  term for  any.
     All  Standard LISP  entities  have some  value unless  an \nameref{error}
     occurs  during evaluation  or the  function causes  transfer  of
     control (such as \nameref{go} and \nameref{return}).
\end{Type}

\begin{Type}{atom}
The set \nameref{any} - \{\nameref{dotted-pair}\}. Any item wich is not a \name{dotted-pair}
is considered as \name{atom}.
\end{Type}


\subsection{Structures}
\begin{Introduction}{Structures}
Structures are  entities created out  of the  primitive types by  the
use of  dotted-pairs.   Lists are  structures very commonly  required
as  actual parameters  to functions.    Where a  list of  homogeneous
entities  is  required by  a  function  this class  will  be  denoted
by <xxx-list>  where xxx  is the  name of  a class  of primitives  or
structures.  Thus a list of ids is an id-list, a list of  integers an
integer-list and so on.
\end{Introduction}

\begin{Concept}{List-Notation}
A  \name{list}  is  recursively  defined  as  \nameref{nil}  or  the  dotted-pair
     (\nameref{any} . list).   A special notation called list-notation is  used
     to represent lists.  List-notation eliminates  extra parentheses
     and dots.   The structure (a .  (b .   (c .  nil))) in list  notation

     is (a b  c).  List-notation and dot-notation may be mixed as  in
     (a b .  c) or (a (b .  c) d) which are (a .  (b .   c)) and (a .
     ((b .   c) .  (d .   nil))). In BNF lists are recognized by  the
     grammar:
\begin{verbatim}
     <left-part> ::= ( | <left-part> <any>
     <list> ::= <left-part>) | <left-part> .  <any>)
\end{verbatim}
     Note:  () is an alternate input representation of nil.
\end{Concept}

\begin{Concept}{alist}
An  association   list;   each  element  of   the  list  is   a
     dotted-pair, the CAR part being a key associated  with the value
     in the CDR part.
\begin{Examples}
((a . 17)(b . (expt x 2))(q . nil))\\
\end{Examples}
Here a is associated with 17 while b is linked to the square of x
and q points to nil. 
\end{Concept}

\begin{Concept}{cond-form}
 A cond-form is a list of 2 element lists of the form:

     (ANTECEDENT:any CONSEQUENT:any)

     The  first element will  henceforth be known  as the  antecedent
     and the  second as the consequent.   The antecedent must have  a
     value.   The consequent may have a value or an occurrence of 
     \nameref{go} or \nameref{return}.
\begin{Examples}
((greaterp x 0) 1)\\
(t 0)\\ 
\end{Examples}
\end{Concept}

\begin{Concept}{lambda}
A  LAMBDA  expression  which  must  have  the  form  (in  list
     notation): 

  (LAMBDA  <parameters>  <body>).  

<parameters>  is  a
     list  of formal parameters  for <body> an  S-expression to  be
     evaluated.   The  semantics of the  evaluation are defined  with
     the  \nameref{eval}.
\begin{Examples}
  (lambda(x y)(cons (car x)(cddr y))) 
\end{Examples}
\end{Concept}

\begin{Concept}{function}

 A LAMBDA expression or a function-pointer to a function.   A
     function is always evaluated as an EVAL, SPREAD form.
(see \nameref{Function Types}).
\end{Concept}



\section{Notation}

\begin{Introduction}{Function Descriptions}

Each function is provided with a prototypical header line. Each formal
parameter is given a name and suffixed with its allowed type.  Lower
case, italic tokens are names of classes and upper case, bold face,
tokens are parameter names referred to in the definition. The type of
the value returned by the function (if any) is suffixed to the
parameter list. 
If it is  not commonly used the parameter  type
may be a specific set enclosed in brackets {...}.  For example:

\begin{verbatim}
PUTD(FNAME:id, TYPE:ftype, BODY:{lambda, function-pointer}):id
\end{verbatim}

PUTD is  a function with three  parameters.   The parameter FNAME  is
an id  to be the  name of the function  being defined.   TYPE is  the
type of the  function being defined and  BODY is a lambda  expression
or a function-pointer.   PUTD returns the name of the function  being
defined.

Functions which  accept formal  parameter lists  of arbitrary  length
have  the  type  class and  parameter  enclosed  in  square  brackets
indicating  that  zero  or more  occurrences  of  that  argument  are
permitted.  For example:

\begin{verbatim}
AND([U:any]):extra-boolean
\end{verbatim}

AND is a function  which accepts zero or more arguments which may  be
of any type.
\end{Introduction}

\begin{Introduction}{Function Types}
\index{eval type}\index{noeval type}
\index{spread type}\index{nospread type}
\index{expr type}\index{macro type}

EVAL  type functions  are  those  which are  invoked  with  evaluated
arguments.  NOEVAL functions are invoked with  unevaluated arguments.
SPREAD  type functions  have  their arguments  passed  in  one-to-one
correspondence  with their  formal parameters.    NOSPREAD  functions
receive their  arguments as a  single list.   EVAL, SPREAD  functions
are  associated  with  EXPRs  and  NOEVAL,  NOSPREAD  functions  with
FEXPRs.     EVAL,  NOSPREAD  and  NOEVAL,  SPREAD  functions  can  be
simulated using NOEVAL, NOSPREAD functions or MACROs.

EVAL, SPREAD  type functions  may have  a maximum  of 15  parameters.
There is  no limit  on the number  of parameters  a NOEVAL,  NOSPREAD
function or MACRO may have.

In the context of  the description of an EVAL, SPREAD function,  then
we  speak of  the  formal parameters  we  mean their  actual  values.
However, in a NOEVAL, NOSPREAD function it is the  unevaluated actual
parameters.

A third function  type, the MACRO, implements functions which  create
S-expressions based  on actual parameters.   When a macro  invocation

is  encountered, the  body  of the  macro,  a lambda  expression,  is
invoked as  a NOEVAL, NOSPREAD function  with the macro's  invocation
bound as  a list to  the macros single  formal parameter.   When  the
macro has been  evaluated the resulting S-expression is  reevaluated.
The  description of  \nameref{eval} and \nameref{expand} provide  precise
details.
\end{Introduction}

\begin{Introduction}{Messages}
\index{error}\index{warning}
Many functions  detect errors.    The description  of such  functions
will  include  these  error  conditions  and  suggested  formats  for
display  of the  generated  error messages.    A  call on  the  
\nameref{error}
function  is  implied  but the  error  number  is  not  specified  by
Standard LISP.  In some cases a  warning message is  sufficient.   To
distinguish between  errors and  warnings, errors  are prefixed  with
five asterisks and warnings with only three.

Primitive  functions  check  arguments that  must  be  of  a  certain
primitive type for  being of that type  and display an error  message
if the  argument is  not correct.    The type  mismatch error  always
takes the form:
\begin{verbatim}
***** PARAMETER not TYPE for FN
\end{verbatim}

Here PARAMETER  is the  unacceptable actual  parameter,  TYPE is  the
type that  PARAMETER was  supposed to  be.   FN  is the  name of  the
function that detected the error.
\end{Introduction}

\begin{Introduction}{Comments}

The character \% signals the start of a comment, text to be ignored during
parsing.  A comment is terminated by the end of the line it is on.  The
function \nameref{readch}must be able to read a comment one character at a
time.  Comments are transparent to the function READ.  The percent sign
may occur as a character in identifiers by preceding it with the escape
character.

   (setq a 17) \% this is a comment

\end{Introduction}

%-----------------------------------------------------------------
\section{Elementary Predicates}
%-----------------------------------------------------------------

\begin{Introduction}{Elementary Predicates}
Functions in this section return \nameref{T} when the condition defined is met
and \nameref{NIL} when it is not. Defined are type checking functions and
elementary comparisons.
\end{Introduction}

%- - - - - - - - - - - - - - - - - - - - - - - - - - - - - - - - - - -
\begin{Function}{atom}
\begin{verbatim}
ATOM(U:any):boolean                       eval, spread
\end{verbatim}
   Returns T if U is not a \nameref{dotted-pair}.
\begin{verbatim}

   EXPR PROCEDURE ATOM(U);
     NULL PAIRP U;
\end{verbatim}
\end{Function}
%- - - - - - - - - - - - - - - - - - - - - - - - - - - - - - - - - - -
\begin{Function}{codep}
\begin{verbatim}
CODEP(U:any):boolean                      eval, spread
\end{verbatim}
   Returns T if U is a \nameref{function-pointer}.
\end{Function}
%- - - - - - - - - - - - - - - - - - - - - - - - - - - - - - - - - - -
\begin{Function}{constantp}
\begin{verbatim}
CONSTANTP(U:any):boolean                  eval, spread
\end{verbatim}
   Returns   T   if  U   is   a  constant   (a  \nameref{number},  
  \nameref{string}, \nameref{function-pointer}, or \nameref{vector}).
\begin{verbatim}

   EXPR PROCEDURE CONSTANTP(U);
     NULL OR(PAIRP U, IDP U);
\end{verbatim}
\end{Function}
%- - - - - - - - - - - - - - - - - - - - - - - - - - - - - - - - - - -
\begin{Function}{eq}
\begin{verbatim}
EQ(U:any, V:any):boolean                  eval, spread
\end{verbatim} 
  Returns  T if U  points to the same  object as V.  EQ is not a
   reliable comparison between numeric arguments.
\end{Function}
%- - - - - - - - - - - - - - - - - - - - - - - - - - - - - - - - - - -
\begin{Function}{eqn}
\begin{verbatim}
EQN(U:any, V:any):boolean                 eval, spread
\end{verbatim}
   Returns  T if U  and V are  EQ or if  U and V  are 
   \nameref{number}s and   have the same value and type.

\end{Function}
%- - - - - - - - - - - - - - - - - - - - - - - - - - - - - - - - - - -
\begin{Function}{equal}
\begin{verbatim}
EQUAL(U:any, V:any):boolean               eval, spread
\end{verbatim}
   Returns  T  if  U  and  V are  the  same.     Dotted-pairs are
   compared  recursively  to the  bottom  levels of  their trees.
   Vectors  must have  identical dimensions  and EQUAL  values in
   all  positions.     Strings  must  have  identical characters.
   Function  pointers must have  \nameref{eq} values.   Other atoms must be
   \nameref{eqn} equal.

\end{Function}
%- - - - - - - - - - - - - - - - - - - - - - - - - - - - - - - - - - -
\begin{Function}{fixp}
\begin{verbatim}
FIXP(U:any):boolean                       eval, spread
\end{verbatim}
   Returns T if U is an \nameref{integer}.
\end{Function}
%- - - - - - - - - - - - - - - - - - - - - - - - - - - - - - - - - - -
\begin{Function}{floatp}
\begin{verbatim}
FLOATP(U:any):boolean                     eval, spread
\end{verbatim}
   Returns T if U is a \nameref{floating} point number.
\end{Function}
%- - - - - - - - - - - - - - - - - - - - - - - - - - - - - - - - - - -
\begin{Function}{idp}
\begin{verbatim}
IDP(U:any):boolean                        eval, spread
\end{verbatim}
   Returns T if U is an \nameref{id}.
\end{Function}
%- - - - - - - - - - - - - - - - - - - - - - - - - - - - - - - - - - -
\begin{Function}{minusp}
\begin{verbatim}
MINUSP(U:any):boolean                     eval, spread
\end{verbatim}
   Returns  T if U is  a number and less  than 0.  If  U is not a
   \nameref{number} or is a positive number, NIL is returned.
\begin{verbatim}

   EXPR PROCEDURE MINUSP(U);
     IF NUMBERP U THEN LESSP(U, 0) ELSE NIL;
\end{verbatim}
\end{Function}
%- - - - - - - - - - - - - - - - - - - - - - - - - - - - - - - - - - -
\begin{Function}{null}
\begin{verbatim}
NULL(U:any):boolean                       eval, spread
\end{verbatim}
   Returns T if U is NIL.
\begin{verbatim}

   EXPR PROCEDURE NULL(U);
     U EQ NIL;
\end{verbatim}
\end{Function}
%- - - - - - - - - - - - - - - - - - - - - - - - - - - - - - - - - - -
\begin{Function}{numberp}
\begin{verbatim}
NUMBERP(U:any):boolean                    eval, spread
\end{verbatim}
   Returns T if U is a \nameref{number}.
\begin{verbatim}

   EXPR PROCEDURE NUMBERP(U);
     IF OR(FIXP U, FLOATP U) THEN T ELSE NIL;
\end{verbatim}
\end{Function}
%- - - - - - - - - - - - - - - - - - - - - - - - - - - - - - - - - - -
\begin{Function}{onep}
\begin{verbatim}
ONEP(U:any):boolean                      eval, spread.
\end{verbatim}
   Returns  T  if U  is a  \nameref{number}  and has  the  value 1  or 1.0.
   Returns NIL otherwise.
\begin{verbatim}


   EXPR PROCEDURE ONEP(U);
     IF EQN(U,1) OR EQN(U,1.0) THEN T ELSE NIL;
\end{verbatim} 
     The  definition in the  published report is  incorrect as it
   does not return T for U of 1.0.

\end{Function}
%- - - - - - - - - - - - - - - - - - - - - - - - - - - - - - - - - - -
\begin{Function}{pairp}
\begin{verbatim}
PAIRP(U:any):boolean                      eval, spread
\end{verbatim}
   Returns T if U is a \nameref{dotted-pair}.

\end{Function}
%- - - - - - - - - - - - - - - - - - - - - - - - - - - - - - - - - - -
\begin{Function}{stringp}
\begin{verbatim}
STRINGP(U:any):boolean                    eval, spread
\end{verbatim}
   Returns T if U is a string.

\end{Function}
%- - - - - - - - - - - - - - - - - - - - - - - - - - - - - - - - - - -
\begin{Function}{vectorp}
\begin{verbatim}
VECTORP(U:any):boolean                    eval, spread
\end{verbatim}
   Returns T if U is a vector.

\end{Function}
%- - - - - - - - - - - - - - - - - - - - - - - - - - - - - - - - - - -
\begin{Function}{zerop}
\begin{verbatim}
ZEROP(U:any):boolean                     eval, spread.
\end{verbatim}
   Returns  T  if U  is a  number  and has  the  value 0  or 0.0.
   Returns NIL otherwise.

     The  definition in the  published report is  incorrect as it
   does not return T for U of 0.0.
\end{Function}

\section{Functions on Dotted-Pairs}
\begin{Introduction}{Function on Dotted-Pairs}
\index{dotted-pair}
The following are elementary functions on dotted-pairs. All functions
in this section which require dotted-pairs as parameters detect a type
mismatch error if the actual parameter is not a dotted-pair.
\end{Introduction}

%- - - - - - - - - - - - - - - - - - - - - - - - - - - - - - - - - - -
\begin{Function}{car}
\begin{verbatim}
CAR(U:dotted-pair):any                    eval, spread
\end{verbatim}
   CAR(CONS(a,  b)) -> a.   The left part of U  is returned.  The
   type mismatch error occurs if U is not a dotted-pair.

\end{Function}
%- - - - - - - - - - - - - - - - - - - - - - - - - - - - - - - - - - -
\begin{Function}{cdr}
\begin{verbatim}
CDR(U:dotted-pair):any                    eval, spread
\end{verbatim}
   CDR(CONS(a,  b)) -> b.  The right  part of U is returned.  The
   type mismatch error occurs if U is not a dotted-pair.

\end{Function}

%- - - - - - - - - - - - - - - - - - - - - - - - - - - - - - - - - - -
\begin{Function}{caar}
\index{CAAAAR}\index{CAAAR}\index{CAAADR}\index{CAADR}
\index{CADR}\index{CAADAR}\index{CADAR}\index{CDAR}\index{CAADDR}
\index{CADDR}\index{CDDR}\index{CADAAR}\index{CDAAR}\index{CADADR}
\index{CDADR}\index{CADDAR}\index{CDDAR}\index{CADDDR}\index{CDDDR}
\index{CDAAAR}\index{CDAADR}\index{CDADAR}\index{CDADDR}\index{CDDAAR}
\index{CDDADR}\index{CDDDAR}\index{CDDDDR}
The composites of CAR and CDR are supported up to 4 levels, namely:

CAAAAR  CAAAR  CAAR CAAADR  CAADR  CADR

CAADAR  CADAR  CDAR CAADDR  CADDR  CDDR

CADAAR  CDAAR  CADADR  CDADR CADDAR  CDDAR

CADDDR  CDDDR  CDAAAR CDAADR CDADAR CDADDR

CDDAAR CDDADR CDDDAR CDDDDR

Here e.g. (cdar x) is equivlaent to (cdr (car x)).

\end{Function}
%- - - - - - - - - - - - - - - - - - - - - - - - - - - - - - - - - - -
\begin{Function}{cons}
\begin{verbatim}
CONS(U:any, V:any):dotted-pair            eval, spread
\end{verbatim}
   Returns  a dotted-pair which  is not \nameref{eq} to  anything and has U
   as its \nameref{car} part and V as its nameref(cdr) part.

\end{Function}
%- - - - - - - - - - - - - - - - - - - - - - - - - - - - - - - - - - -
\begin{Function}{list}
\begin{verbatim}
LIST([U:any]):list            noeval, nospread, or macro
\end{verbatim}
   A  list of  the evaluation of  each element of  U is returned.
   The  order  of evaluation  nead not  be first  to last  as the
   following definition implies.
\begin{verbatim}

   FEXPR PROCEDURE LIST(U);
       EVLIS U;
\end{verbatim}
     The   published  report's  definition   implies  a  specific
   ordering.
\end{Function}
%- - - - - - - - - - - - - - - - - - - - - - - - - - - - - - - - - - -
\begin{Function}{rplaca}
\begin{verbatim}
RPLACA(U:dotted-pair, V:any):dotted-pair  eval, spread
\end{verbatim}
   The  \nameref{car} portion  of the  dotted-pair U  is replaced by  V. If
   dotted-pair U  is (a .  b) then (V . b) is returned.  The type
   mismatch error occurs if U is not a dotted-pair.
\end{Function}
%- - - - - - - - - - - - - - - - - - - - - - - - - - - - - - - - - - -
\begin{Function}{rplacd}
\begin{verbatim}
RPLACD(U:dotted-pair, V:any):dotted-pair  eval, spread
\end{verbatim}
   The  \nameref{cdr} portion  of the  dotted-pair U  is replaced by  V. If
   dotted-pair  U is (a  .  b)  then (a .   V) is  returned.  The
   type mismatch error occurs if U is not a dotted-pair.

\end{Function}
%- - - - - - - - - - - - - - - - - - - - - - - - - - - - - - - - - - -

\section{Functions for Identifiers}

\begin{Concept}{oblist}
The following functions deal with identifiers and the \nameref{oblist},
the structure of which is not defined.
The \name{oblist} is an internal stucture where \nameref{id}s
are kept.  The function of the \name{oblist} is
to provide a symbol table for identifiers created during input.
Identifiers created by \nameref{read} which have the same characters will
therefore refer to the same object (see the \nameref{eq} function).

Identifiers created by \nameref{gensym} or \nameref{compress} are
not member of the \name{oblist} and therefore they are not 
unique even if they are represented by the same character
sequence on output. The function \nameref{intern} is used
to create an equivalent unique \name{id} which then is
member of the \name{oblist}.
\end{Concept}


%- - - - - - - - - - - - - - - - - - - - - - - - - - - - - - - - - - -
\begin{Function}{compress}
\begin{verbatim}
COMPRESS(U:id-list):{atom-vector}         eval, spread
\end{verbatim}
   U  is a  list of single  character identifiers  which is built
   into  a Standard  LISP entity  and returned.    Recognized are
   \nameref{number}s,  \nameref{string}s, 
   and identifiers (see  \nameref{id}) with  the \name{escape} character
   prefixing  special  characters.    Function pointers
   may  be compressed but this is an undefined use.  If an entity
   cannot  be parsed out  of U or characters  are left over after
   parsing an error occurs:
\begin{verbatim}
   ***** Poorly formed atom in COMPRESS
\end{verbatim}
\end{Function}
%- - - - - - - - - - - - - - - - - - - - - - - - - - - - - - - - - - -

\begin{Function}{explode}
\begin{verbatim}
EXPLODE(U:{atom}-{vector}):id-list        eval, spread
\end{verbatim}
   Returned  is a  list of  interned characters  representing the
   characters  to print  of the  value of  U. The  primitive data
   types have these formats:

   \nameref{integer}: Leading   zeroes  are  suppressed  and  a  minus  sign
         prefixes the digits if the integer is negative.

   \nameref{floating}: The  value appears  in the  format [-]0.nn...nnE[-]mm
         if the magnitude of the number  is too large or small to
         display in [-]nnnn.nnnn format.   The crossover point is
         determined by the implementation.

   \nameref{id}: The   characters  of  the  print  name  of  the  identifier
         are produced with  special characters  prefixed with the
         escape character.

   \nameref{string}: The  characters of  the string  are produced surrounded
         by double quotes "...".

   \nameref{function-pointer}: The  value of the function-pointer is created
         as a list of characters conforming to the conventions of
         the system site.

   The  type  mismatch  error  occurs  if  U  is  not  a  number,
   identifier, string, or function-pointer.

\end{Function}
%- - - - - - - - - - - - - - - - - - - - - - - - - - - - - - - - - - -
\begin{Function}{gensym}
\begin{verbatim}
GENSYM():identifier                       eval, spread
\end{verbatim}
   Creates  an identifier which is not interned on the \nameref{oblist} and
   consequently not \nameref{eq} to anything else.

\end{Function}
%- - - - - - - - - - - - - - - - - - - - - - - - - - - - - - - - - - -
\begin{Function}{intern}
\begin{verbatim}
INTERN(U:{id,string}):id                  eval, spread
\end{verbatim}
   INTERN  searches the  \nameref{oblist} for  an identifier  with the same
   print  name as U and returns the identifier on the \name{oblist} if a
   match  is found.  Any  properties and global values associated
   with  U  may be  lost.    If U  does  not match  any  entry, a
   new  one is  created and  returned.   If  U has more  than the
   maximum  number of characters  permitted by the implementation
   (the minimum number is 24) an error occurs:
\begin{verbatim}
   ***** Too many characters to INTERN
\end{verbatim}
\end{Function}
%- - - - - - - - - - - - - - - - - - - - - - - - - - - - - - - - - - -
\begin{Function}{remob}
\begin{verbatim}
REMOB(U:id):id                            eval, spread
\end{verbatim}
   If  U is present on the  \nameref{oblist} it is removed.   This does not
   affect  U having properties, flags, functions and the like.  U
   is returned.

\end{Function}
%- - - - - - - - - - - - - - - - - - - - - - - - - - - - - - - - - - -

\section{Property List Functions}
\begin{Introduction}{Property List Functions}
With each id in the system is a \name{property list}, a set of entities
which are associated with the id for fast access. These entities are
called \nameindex{flags} if their use gives the id a single valued
property, and \nameindex{properties} if the id is to have a multivalued
attribute: an indicator with a property.

Flags and indicators may clash, consequently care should be taken to
avoid this occurrence. Flagging X with an id which already is an
indicator for X may result in that indicator and associated property
being lost. Likewise, adding an indicator which is the same id as a
flag may result in the flag being destroyed.
\end{Introduction}


%- - - - - - - - - - - - - - - - - - - - - - - - - - - - - - - - - - -
\begin{Function}{flag}
\begin{verbatim}
FLAG(U:id-list, V:id):NIL                 eval, spread
\end{verbatim}
   U  is a list  of ids which  are flagged with V.  The effect of
   \name{flag}  is that \nameref{flagp} will  have the value T for  those ids of U
   which  were flagged.  Both V and all the elements of U must be
   identifiers or the type mismatch error occurs.
\begin{Examples}
flag('(u v),'symmetric)\\
\end{Examples}
Note: If you want to flag a single \name{id} you must put it into
a list before calling the function \name{flag}. A flag is removed
by \nameref{remflag}
\end{Function}
%- - - - - - - - - - - - - - - - - - - - - - - - - - - - - - - - - - -
\begin{Function}{flagp}
\begin{verbatim}
FLAGP(U:any, V:any):boolean               eval, spread
\end{verbatim}
   Returns  T if U has been  previously flagged (see \nameref{flag}}
   with V, else NIL. Returns NIL if either U or V is not an \nameref{id}.


\end{Function}
%- - - - - - - - - - - - - - - - - - - - - - - - - - - - - - - - - - -
\begin{Function}{get}
\begin{verbatim}
GET(U:any, IND:any):any                   eval, spread
\end{verbatim}
   Returns  the property  associated with indicator  IND from the
   property  list of U. If U does  not have indicator IND, NIL is
   returned.   GET  cannot be used to  access functions (use GETD
   instead). For setting a property use the function \nameref{put}.


\end{Function}
%- - - - - - - - - - - - - - - - - - - - - - - - - - - - - - - - - - -
\begin{Function}{put}
\begin{verbatim}
PUT(U:id, IND:id, PROP:any):any           eval, spread
\end{verbatim}
   The  indicator IND  with the  property PROP  is placed  on the
   property  list of the id  U. If the action  of PUT occurs, the
   value  of PROP is  returned.  If  either of U  and IND are not
   ids  the type mismatch  error will occur  and no property will
   be  placed.  PUT cannot be  used to define functions
  (use \nameref{putd} instead). The values stored on the property
   list can be retrieved using \nameref{get}. \nameref{remprop}
   removes a property.
\begin{Examples}
put('otto,'hugo,'(a))\\
get('otto,'hugo) & (a)\\
put('otto,'hugo,'(b))\\
get('otto,'hugo) & (b)\\
remprop('otto,'hugo)\\
get('otto,'hugo) & nil\\
\end{Examples}
\end{Function}
%- - - - - - - - - - - - - - - - - - - - - - - - - - - - - - - - - - -
\begin{Function}{remflag}
\begin{verbatim}
REMFLAG(U:any-list, V:id):NIL             eval, spread
\end{verbatim}
   Removes  the flag V  from the property list  of each member of
   the  list U. Both V and  all the elements of  U must be ids or
   the type mismatch error will occur (see \nameref{flag}).

\end{Function}
%- - - - - - - - - - - - - - - - - - - - - - - - - - - - - - - - - - -
\begin{Function}{remprop}
\begin{verbatim}
REMPROP(U:any, IND:any):any               eval, spread
\end{verbatim}
   Removes  the  property with  indicator  IND from  the property
   list  of U. Returns  the removed property or  NIL if there was
   no such indicator (see \nameref{put}}.
\end{Function}

%- - - - - - - - - - - - - - - - - - - - - - - - - - - - - - - - - - -
%- - - - - - - - - - - - - - - - - - - - - - - - - - - - - - - - - - -

\section{Function Definition}
\begin{Introduction}{Function Definition}
Functions in Standard LISP are global entities. To avoid
function-variable naming clashes no variable may have the same name as
a function. 
\end{Introduction}

%- - - - - - - - - - - - - - - - - - - - - - - - - - - - - - - - - - -
\begin{Function}{de}
\index{expr}
\begin{verbatim}
DE(FNAME:id, PARAMS:id-list, FN:any):id      noeval, nospread
\end{verbatim}
   The  function  FN with  the  formal parameter  list  PARAMS is
   added  to the  set of defined  functions with  the name FNAME.
   Any  previous  definitions  of the  function  are lost.    The
   function  created is of type EXPR (see \nameref{Function Types}). If  \nameref{*COMP} is
   non-NIL,  the EXPR is first compiled.  The name of the defined
   function is returned.
\begin{verbatim}

   FEXPR PROCEDURE DE(U);
     PUTD(CAR U, 'EXPR, LIST('LAMBDA, CADR U, CADDR U));
\end{verbatim}
\end{Function}
%- - - - - - - - - - - - - - - - - - - - - - - - - - - - - - - - - - -
\begin{Function}{df}
\index{fexpr}
\begin{verbatim}
DF(FNAME:id, PARAM:id-list, FN:any):id       noeval, nospread
\end{verbatim}
   The  function FN with  formal parameter PARAM  is added to the
   set  of defined  functions with  the name  FNAME. Any previous
   definitions  of the function  are lost.   The function created
   is  of  type FEXPR (see \nameref{Function Types}).  If  \nameref{*COMP} is  T  the FEXPR
   is  first  compiled.    The name  of the  defined  function is
   returned.
\begin{verbatim}

   FEXPR PROCEDURE DF(U);
     PUTD(CAR U, 'FEXPR, LIST('LAMBDA, CADR U, CADDR U));
\end{verbatim}

\end{Function}
%- - - - - - - - - - - - - - - - - - - - - - - - - - - - - - - - - - -
\begin{Function}{dm}
\index{macro}
\begin{verbatim}
DM(MNAME:id, PARAM:id-list, FN:any):id       noeval, nospread
\end{verbatim}
   The  macro FN with the formal  parameter PARAM is added to the
   set  of defined  functions with  the name  MNAME. Any previous
   definitions  of the  function are  overwritten.   The function
   created is of type MACRO (see \nameref{Function Types}).
   The name of the macro is returned.
\begin{verbatim}

   FEXPR PROCEDURE DM(U);
     PUTD(CAR U, 'MACRO, LIST('LAMBDA, CADR U, CADDR U));
\end{verbatim}


\end{Function}
%- - - - - - - - - - - - - - - - - - - - - - - - - - - - - - - - - - -
\begin{Function}{getd}
\begin{verbatim}
GETD(FNAME:any):{NIL, dotted-pair}        eval, spread
\end{verbatim}
   If   FNAME  is  not  the  name  of  a  defined  function,  NIL
   is  returned.     If  FNAME is  a  defined  function  then the
   dotted-pair
\begin{verbatim}
   (TYPE:ftype .  DEF:{function-pointer, lambda})
\end{verbatim}
   is returned.

\end{Function}
%- - - - - - - - - - - - - - - - - - - - - - - - - - - - - - - - - - -
\begin{Function}{putd}
\begin{verbatim}
PUTD(FNAME:id, TYPE:ftype, BODY:function):id             eval, spread
\end{verbatim}
   Creates  a  function with  name FNAME  and definition  BODY of
   type  TYPE. If PUTD succeeds the  name of the defined function
   is  returned.  The  effect of PUTD is that  GETD will return a
   dotted-pair  with the functions type and definition.  Likewise
   the  \nameref{globalp} predicate will  return T  when queried  with the
   function name.
   If  the function FNAME  has already been  declared as a GLOBAL
   or FLUID variable the error:
\begin{verbatim}
   ***** FNAME is a non-local variable
\end{verbatim}
   occurs  and the  function will  not be  defined.   If function
   FNAME already exists a warning message will appear:
\begin{verbatim}
   *** FNAME redefined
\end{verbatim}
   The   function  defined  by  PUTD   will  be  compiled  before
   definition if \nameref{*COMP} variable is non-NIL.


\end{Function}
%- - - - - - - - - - - - - - - - - - - - - - - - - - - - - - - - - - -
\begin{Function}{remd}
\begin{verbatim}
REMD(FNAME:id):{NIL, dotted-pair}         eval, spread
\end{verbatim}
   Removes  the  function named  FNAME  from the  set  of defined
   functions.    Returns the (ftype  .   function) dotted-pair or
   NIL  as does \nameref)getd}.  The global/function attribute  of FNAME is
   removed and the name may be used subsequently as a variable.
\end{Function}
%- - - - - - - - - - - - - - - - - - - - - - - - - - - - - - - - - - -


\section{Variables and Bindings}

\begin{Introduction}{Scope}
\index{variables}
A variable is a place holder for a Standard LISP entity which is said
to be bound to the variable. The scope of a variable is the range over
which the variable has a defined value. There are three different
binding mechanisms in Standard LISP: 
\nameref{Local Binding}, \nameref{Global Binding}, and
\nameref{Fluid Binding}.
\end{Introduction}

\begin{Concept}{Local Binding}
\index{variables}
This type of binding occurs
only in compiled functions. Local variables occur as formal parameters
in \nameref{lambda} expressions (function arguments)
 and as \nameref{prog} form variables. The binding occurs
when a lambda expression is evaluated or when a \name{prog} form is executed.
The scope of a local variable is the body of the function in which it
is defined.
\end{Concept}

\begin{Concept}{Global Binding}
\index{variables}
Only one binding of a 
global variable exists at any time allowing direct access to the value
bound to the variable.  The scope of a global variable is universal.
Variables declared \nameref{global} may not appear as parameters 
in \nameref{lambda} expressions (function arguments)
 or as \nameref{prog} form variables. A variable must be declared
\name{global} prior to its use as a global variable since the default type
for undeclared variables is \nameref{fluid}.
\end{Concept}



\begin{Concept}{Fluid Binding}
\index{variables}
Fluid variables are global
in scope but may occur as \name{fluid} formal parameters or
\nameref{prog} form variables. In interpreted functions all formal parameters
and \name{prog} form variables are considered to have fluid binding until
changed to local binding by compilation.  When \name{fluid} variables are
used as parameters (\nameref{lambda} expressions}
they are rebound in such a way that the previous
binding may be restored. All references to \name{fluid} variables are to the
currently active binding.
\end{Concept}

%- - - - - - - - - - - - - - - - - - - - - - - - - - - - - - - - - - -
\begin{Function}{fluid}
\index{variables}
\begin{verbatim}
FLUID(IDLIST:id-list):NIL                 eval, spread
\end{verbatim}
   The  ids in IDLIST  are declared as  FLUID type variables (ids
   not  previously  declared are  initialized to  NIL). Variables
   in  IDLIST already  declared FLUID  are ignored.    Changing a
   variable's  type from GLOBAL  to FLUID is  not permissible and
   results in the error:
\begin{verbatim}
   ***** ID cannot be changed to FLUID
\end{verbatim}
\end{Function}
%- - - - - - - - - - - - - - - - - - - - - - - - - - - - - - - - - - -
\begin{Function}{fluidp}
\index{variables}
\begin{verbatim}
FLUIDP(U:any):boolean                     eval, spread
\end{verbatim}
   If  U  has  been declared by \nameref{fluid} T is
   returned, otherwise NIL is returned.

\end{Function}
%- - - - - - - - - - - - - - - - - - - - - - - - - - - - - - - - - - -
\begin{Function}{global}
\index{variables}
\begin{verbatim}
GLOBAL(IDLIST:id-list):NIL                eval, spread
\end{verbatim}
   The  ids of  IDLIST are  declared global  type variables.   If
   an  id has not  been declared previously  it is initialized to
   NIL.  Variables already declared GLOBAL are ignored.  Changing
   a  variables type from FLUID to  GLOBAL is not permissible and
   results in the error:
\begin{verbatim}
   ***** ID cannot be changed to GLOBAL
\end{verbatim}
\end{Function}
%- - - - - - - - - - - - - - - - - - - - - - - - - - - - - - - - - - -
\begin{Function}{globalp}
\index{variables}
\begin{verbatim}
GLOBALP(U:any):boolean                    eval, spread
\end{verbatim}
   If  U has  been declared  GLOBAL or is  the name  of a defined
   function, T is returned, else NIL is returned.
\end{Function}
%- - - - - - - - - - - - - - - - - - - - - - - - - - - - - - - - - - -
\begin{Function}{set}
\index{variables}
\begin{verbatim}
SET(EXP:id, VALUE:any):any                eval, spread
\end{verbatim}
   EXP  must be  an identifier or  a type  mismatch error occurs.
   The  effect  of  SET  is  replacement  of  the  item bound  to
   the  identifier by  VALUE. If  the identifier  is not  a local
   variable  or has not been  declared GLOBAL it is automatically
   declared FLUID with the resulting warning message:
\begin{verbatim}
   *** EXP declared FLUID
\end{verbatim}
   EXP must not evaluate to T or NIL or an error occurs:
\begin{verbatim}
   ***** Cannot change T or NIL
\end{verbatim}
\end{Function}
%- - - - - - - - - - - - - - - - - - - - - - - - - - - - - - - - - - -
\begin{Function}{setq}
\index{variables}
\begin{verbatim}
SETQ(VARIABLE:id, VALUE:any):any          noeval, nospread
\end{verbatim}
   If  VARIABLE is not local or  GLOBAL it is by default declared
   FLUID and the warning message:
\begin{verbatim}
   *** VARIABLE declared FLUID
\end{verbatim}
   appears.    The value  of the  current binding of  VARIABLE is
   replaced  by the value of VALUE. VARIABLE must not be T or NIL
   or an error occurs:
\begin{verbatim}
   ***** Cannot change T or NIL
\end{verbatim}

\begin{verbatim}
   MACRO PROCEDURE SETQ(X);
     LIST('SET, LIST('QUOTE, CADR X), CADDR X);
\end{verbatim}
\end{Function}
%- - - - - - - - - - - - - - - - - - - - - - - - - - - - - - - - - - -
\begin{Function}{unfluid}
\index{variables}
\begin{verbatim}
UNFLUID(IDLIST:id-list):NIL               eval, spread
\end{verbatim}
   The  variables  in IDLIST  that  have been  declared  as \nameref{fluid}
   variables   are  no  longer  considered  as  fluid  variables.
   Others  are  ignored.   This  affects only  compiled functions
   as  free variables in  interpreted functions are automatically
   considered fluid.

\end{Function}
%- - - - - - - - - - - - - - - - - - - - - - - - - - - - - - - - - - -



\section{Program Feature Functions}


%- - - - - - - - - - - - - - - - - - - - - - - - - - - - - - - - - - -
\begin{Function}{go}
\index{program control}
\index{label}
\begin{verbatim}
GO(LABEL:id)                               noeval, nospread
\end{verbatim}
   GO  alters the normal flow of  control within a \nameref{prog} function.
   The  next  statement of  a PROG  function  to be  evaluated is
   immediately  preceded by  LABEL. A GO  may only  appear in the
   following situations:

      1. At the  top level  of a \nameref{prog}  referencing a  label which
         also appears at the top level of the same prog.

      2. As the consequent of a \nameref{cond} item of a \name{cond} appearing on
         the top level of a \nameref{prog}.

      3. As the consequent  of a \nameref{cond} item which  appears as the
         consequent of a \name{cond} item to any level.

      4. As  the last  statement  of  a  \nameref{progn}  which  appears at
         the top  level of  a  \nameref{prog} or  in  a \name{progn}  appearing in
         the consequent  of a  \nameref(cond} to any  level subject  to the
         restrictions of 2 and 3.

      5. As the last  statement of a \nameref{progn} 
          within a \name{progn} or as
         the consequent of a \nameref{prog}in a \name{progn}to any level subject
         to the restrictions of 2, 3 and 4.

   If  LABEL does  not appear  at the  top level  of the  \name{prog} in
   which the \name{go} appears, an error occurs:
\begin{verbatim}
   ***** LABEL is not a known label
\end{verbatim}

   If  the \name{go} has been placed in  a position not defined by rules
   1-5, another error is detected:
\begin{verbatim}
   ***** Illegal use of GO to LABEL
\end{verbatim}
\end{Function}
%- - - - - - - - - - - - - - - - - - - - - - - - - - - - - - - - - - -
\begin{Function}{prog}
\index{program control}
\begin{verbatim}
PROG(VARS:id-list, [PROGRAM:{id, any}]):any     noeval, nospread
\end{verbatim}
   VARS  is a  list of  ids which  are considered fluid  when the
   PROG  is interpreted and local  when compiled (see ``Variables
   and  Bindings'', section 3.6 on page 22).  The PROGs variables
   are  allocated space  when the  PROG form  is invoked  and are
   deallocated  when  the PROG  is exited.    PROG  variables are
   initialized  to NIL. The PROGRAM is a set of expressions to be
   evaluated  in order of their  appearance in the PROG function.
   Identifiers  appearing  in the  top level  of the  PROGRAM are
   labels  which can be  referenced by GO.  The value returned by
   the  PROG function is  determined by a  \nameref{return} function or NIL
   if the PROG falls through.

\end{Function}
%- - - - - - - - - - - - - - - - - - - - - - - - - - - - - - - - - - -
\begin{Function}{progn}
\index{program control}
\begin{verbatim}
PROGN([U:any]):any                              noeval, nospread
\end{verbatim}
   U  is a  set of  expressions which  are executed sequentially.
   The value returned is the value of the last expression.
\end{Function}
%- - - - - - - - - - - - - - - - - - - - - - - - - - - - - - - - - - -
\begin{Function}{prog2}
\index{program control}
\begin{verbatim}
PROG2(A:any, B:any)any                    eval, spread
\end{verbatim}
   Returns the value of B.
\begin{verbatim}

   EXPR PROCEDURE PROG2(A, B);
     B;
\end{verbatim}
\end{Function}
%- - - - - - - - - - - - - - - - - - - - - - - - - - - - - - - - - - -
\begin{Function}{return}
\index{program control}
\begin{verbatim}
RETURN(U:any)                             eval, spread
\end{verbatim}
   Within  a  \nameref{prog}, RETURN  terminates the  evaluation of  a PROG
   and  returns U as  the value of the  PROG. The restrictions on
   the  placement  of RETURN  are exactly  those of nameref{go}. Improper
   placement of RETURN results in the error:
\begin{verbatim}
   ***** Illegal use of RETURN
\end{verbatim}
\end{Function}
%- - - - - - - - - - - - - - - - - - - - - - - - - - - - - - - - - - -


\section{Error Handling}

%- - - - - - - - - - - - - - - - - - - - - - - - - - - - - - - - - - -
\begin{Function}{error}
\index{error handling}
\begin{verbatim}
ERROR(NUMBER:integer, MESSAGE:any)        eval, spread
\end{verbatim}
   NUMBER  and MESSAGE are passed  back to a surrounding \nameref{errorset}
   (the  Standard LISP reader has an ERRORSET). MESSAGE is placed
   in  the global  variable \nameref{emsg*}  and the  error number becomes
   the  value of  the surrounding  ERRORSET. \nameref{fluid}  variables and
   local  bindings  are  unbound  to  return  to  the environment
   of  the  ERRORSET. Global  variables are  not affected  by the
   process.

\end{Function}
%- - - - - - - - - - - - - - - - - - - - - - - - - - - - - - - - - - -
\begin{Function}{errorset}
\index{error handling}
\begin{verbatim}
ERRORSET(U:any, MSGP:boolean, TR:boolean):any            eval, spread
\end{verbatim}
   If  an  error occurs  during the  evaluation  of U,  the value
   of  NUMBER from  the \nameref{error}  call is  returned as the  value of
   ERRORSET.  In  addition,  if  the value  of  MSGP  is non-NIL,
   the  MESSAGE from  the ERROR call  is displayed  upon both the
   standard  output  device  and  the  currently  selected output
   device  unless the  standard output device  is not open.   The
   message  appears prefixed with 5 asterisks.   The MESSAGE list
   is  displayed  without top  level  parentheses.    The MESSAGE
   from  the ERROR call will be  available in the global variable
   \nameref{emsg*}.    The  exact format  of  error messages  generated by
   Standard  LISP functions  described in  this document  are not
   fixed  and should not  be relied upon to  be in any particular
   form.    Likewise,  error numbers  generated by  Standard LISP
   functions are implementation dependent.
   If  no error occurs  during the evaluation of  U, the value of
   (LIST (EVAL U)) is returned.

   If  an error has been signaled and  the value of TR is non-NIL
   a  traceback sequence will be initiated on the selected output
   device.     The  traceback will  display  information  such as
   unbindings  of \nameref{fluid} variables, argument lists and so on in an
   implementation dependent format.

\end{Function}
%- - - - - - - - - - - - - - - - - - - - - - - - - - - - - - - - - - -

\section{Functions for Vectors}


%- - - - - - - - - - - - - - - - - - - - - - - - - - - - - - - - - - -
\begin{Function}{getv}
\index{vector}
\begin{verbatim}
GETV(V:vector, INDEX:integer):any         eval, spread
\end{verbatim}
   Returns  the value stored  at position INDEX  of the \nameref{vector} V.
   The  type mismatch error  may occur.   An error  occurs if the
   INDEX does not lie within 0... UPBV(V) inclusive:
\begin{verbatim}
   ***** INDEX subscript is out of range
\end{verbatim}
\end{Function}
%- - - - - - - - - - - - - - - - - - - - - - - - - - - - - - - - - - -
\begin{Function}{mkvect}
\index{vector}
\begin{verbatim}
MKVECT(UPLIM:integer):vector              eval, spread
\end{verbatim}
   Defines   and  allocates  space  for  a  \nameref{vector}  with  UPLIM+1
   elements  accessed as 0... UPLIM. Each  element is initialized
   to  NIL. An error will occur  if UPLIM is <  0 or there is not
   enough space for a vector of this size:
\begin{verbatim}
   ***** A vector of size UPLIM cannot be allocated
\end{verbatim}
\end{Function}
%- - - - - - - - - - - - - - - - - - - - - - - - - - - - - - - - - - -
\begin{Function}{putv}
\index{vector}
\begin{verbatim}
PUTV(V:vector, INDEX:integer, VALUE:any):any             eval, spread
\end{verbatim}
   Stores  VALUE into  the \nameref{vector} V  at position  INDEX. VALUE is
   returned.   The type mismatch error may  occur.  If INDEX does
   not lie in 0... UPBV(V) an error occurs:
\begin{verbatim}
   ***** INDEX subscript is out of range
\end{verbatim}
\end{Function}
%- - - - - - - - - - - - - - - - - - - - - - - - - - - - - - - - - - -
\begin{Function}{upbv}
\index{vector}
\begin{verbatim}
UPBV(U:any):NIL,integer                   eval, spread
\end{verbatim}
   Returns  the upper limit of U  if U is a  \nameref{vector}, or NIL if it
   is not.

\end{Function}
%- - - - - - - - - - - - - - - - - - - - - - - - - - - - - - - - - - -


\section{Boolean Functions, Conditionals}

%- - - - - - - - - - - - - - - - - - - - - - - - - - - - - - - - - - -
\begin{Function}{and}
\index{boolean}
\begin{verbatim}
AND([U:any]):extra-boolean                    noeval, nospread
\end{verbatim}
   AND  evaluates each U until a value of NIL is found or the end
   of  the list is encountered.   If a non-NIL  value is the last
   value it is returned, or NIL is returned.
\begin{verbatim}

   FEXPR PROCEDURE AND(U);
   BEGIN
     IF NULL U THEN RETURN NIL;
   LOOP: IF NULL CDR U THEN RETURN EVAL CAR U
            ELSE IF NULL EVAL CAR U THEN RETURN NIL;
      U := CDR U;
      GO LOOP
   END;
\end{verbatim}

\end{Function}
%- - - - - - - - - - - - - - - - - - - - - - - - - - - - - - - - - - -
\begin{Function}{cond}
\index{boolean}
\begin{verbatim}
COND([U:cond-form]):any                       noeval, nospread
\end{verbatim}
   The  antecedents of all U's (\nameref{cond-form}s) are  evaluated in  order of their
   appearance  until  a  non-NIL  value  is  encountered.     The
   consequent  of  the selected  U is  evaluated and  becomes the
   value  of  the  COND.  The  consequent  may  also  contain the
   special  functions \nameref{go} and \nameref{return}  subject to  the restraints
   given  for these  functions.   In these cases COND  does not have
   a  defined value, but rather  an effect.   If no antecedent is
   non-NIL  the value of COND is NIL. An error is detected if a U
   is improperly formed:
\begin{verbatim}
   ***** Improper cond-form as argument of COND
\end{verbatim}
\end{Function}
%- - - - - - - - - - - - - - - - - - - - - - - - - - - - - - - - - - -
\begin{Function}{not}
\index{boolean}
\begin{verbatim}
NOT(U:any):boolean                        eval, spread
\end{verbatim}
   If  U  is NIL,  return  T else  return NIL  (same  as function
   NULL).
\begin{verbatim}

   EXPR PROCEDURE NOT(U);
     U EQ NIL;
\end{verbatim}
\end{Function}
%- - - - - - - - - - - - - - - - - - - - - - - - - - - - - - - - - - -
\begin{Function}{or}
\index{boolean}
\begin{verbatim}
OR([U:any]):extra-boolean                  noeval, nospread
\end{verbatim}
   U  is any number  of expressions which  are evaluated in order
   of  their appearance.   When one is found to  be non-NIL it is
   returned as the value of OR. If all are NIL, NIL is returned.
\begin{verbatim}

   FEXPR PROCEDURE OR(U);
   BEGIN SCALAR X;
   LOOP: IF NULL U THEN RETURN NIL
            ELSE IF (X := EVAL CAR U) THEN RETURN X;
      U := CDR U;
      GO LOOP
   END;
\end{verbatim}
\end{Function}
%- - - - - - - - - - - - - - - - - - - - - - - - - - - - - - - - - - -


\section{Arithmetic Functions}
\begin{Introduction}{Conversion}
\index{mixed-mode arithmetic}
Conversions between numeric types are provided explicitly by the
\nameref{fix} and \nameref{float} functions and implicitly by any 
multi-parameter arithmetic function which receives mixed types of arguments. A
conversion from fixed to floating point numbers may result in a loss
of precision without a warning message being generated. Since
integers may have a greater magnitude that that permitted for floating
numbers, an error may be signaled when the attempted conversion cannot
be done.

 Because the magnitude of integers is unlimited the conversion
of a floating point number to a fixed number is always possible, the
only loss of precision being the digits to the right of the decimal
point which are truncated. If a function receives mixed types of
arguments the general rule will have the fixed numbers converted to
floating before arithmetic operations are performed. In all cases an
error occurs if the parameter to an arithmetic function is not a
number:
\begin{verbatim}
\errormessage{***** XXX parameter to FUNCTION is not a number}
\end{verbatim}
XXX is the value of the parameter at fault and FUNCTION is the name of
the function that detected the error. Exceptions to the rule are noted
where they occur.
\end{Introduction}


%- - - - - - - - - - - - - - - - - - - - - - - - - - - - - - - - - - -
\begin{Function}{abs}
\index{arithmetic}
\begin{verbatim}
ABS(U:number):number                      eval, spread
\end{verbatim}
   Returns the absolute value of its argument.
\begin{verbatim}

   EXPR PROCEDURE ABS(U);
     IF LESSP(U, 0) THEN MINUS(U) ELSE U;
\end{verbatim}

\end{Function}
%- - - - - - - - - - - - - - - - - - - - - - - - - - - - - - - - - - -
\begin{Function}{add1}
\index{arithmetic}
\begin{verbatim}
ADD1(U:number):number                     eval, spread
\end{verbatim}
   Returns  the value of U plus 1 of the same type as U (fixed or
   floating).
\begin{verbatim}

   EXPR PROCEDURE ADD1(U);
     PLUS2(U, 1);
\end{verbatim}

\end{Function}
%- - - - - - - - - - - - - - - - - - - - - - - - - - - - - - - - - - -
\begin{Function}{difference}
\index{arithmetic}
\begin{verbatim}
DIFFERENCE(U:number, V:number):number     eval, spread
\end{verbatim}
   The value U - V is returned.

\end{Function}
%- - - - - - - - - - - - - - - - - - - - - - - - - - - - - - - - - - -
\begin{Function}{divide}
\index{arithmetic}
\begin{verbatim}
DIVIDE(U:number, V:number):dotted-pair    eval, spread
\end{verbatim}
   The  dotted-pair (quotient  .   remainder)  is returned.   The
   quotient  part is  computed the  same as  by QUOTIENT  and the
   remainder  the  same  as  by  REMAINDER.  An  error  occurs if
   division by zero is attempted:
\begin{verbatim}
   ***** Attempt to divide by 0 in DIVIDE
\end{verbatim}
\begin{verbatim}

   EXPR PROCEDURE DIVIDE(U, V);
     (QUOTIENT(U, V) . REMAINDER(U, V));
\end{verbatim}

\end{Function}
%- - - - - - - - - - - - - - - - - - - - - - - - - - - - - - - - - - -
\begin{Function}{expt}
\index{arithmetic}
\begin{verbatim}
EXPT(U:number, V:integer):number          eval, spread
\end{verbatim}
   Returns  U raised to  the V power.   A floating  point U to an
   integer  power V does not have  V changed to a floating number
   before exponentiation.

\end{Function}
%- - - - - - - - - - - - - - - - - - - - - - - - - - - - - - - - - - -
\begin{Function}{fix}
\index{arithmetic}
\begin{verbatim}
FIX(U:number):integer                     eval, spread
\end{verbatim}
   Returns  an integer  which corresponds to  the truncated value
   of  U. The  result of  conversion must  retain all significant
   portions of U. If U is an integer it is returned unchanged.

\end{Function}
%- - - - - - - - - - - - - - - - - - - - - - - - - - - - - - - - - - -
\begin{Function}{float}
\index{arithmetic}
\begin{verbatim}
FLOAT(U:number):floating                  eval, spread
\end{verbatim}
   The  floating point number  corresponding to the  value of the
   argument  U  is  returned.    Some  of  the  least significant
   digits  of an integer may be  lost do to the implementation of
   floating  point numbers.    FLOAT of  a floating  point number
   returns  the number unchanged.  If U is too large to represent
   in floating point an error occurs:
\begin{verbatim}
   ***** Argument to FLOAT is too large
\end{verbatim}
\end{Function}
%- - - - - - - - - - - - - - - - - - - - - - - - - - - - - - - - - - -
\begin{Function}{greaterp}
\index{arithmetic}\index{boolean}
\begin{verbatim}
GREATERP(U:number, V:number):boolean      eval, spread
\end{verbatim}
   Returns  T if U is strictly  greater than V, otherwise returns
   NIL.

\end{Function}
%- - - - - - - - - - - - - - - - - - - - - - - - - - - - - - - - - - -
\begin{Function}{lessp}
\index{arithmetic}\index{boolean}
\begin{verbatim}
LESSP(U:number, V:number):boolean         eval, spread
\end{verbatim}
   Returns  T if  U is  strictly less  than V,  otherwise returns
   NIL.

\end{Function}
%- - - - - - - - - - - - - - - - - - - - - - - - - - - - - - - - - - -
\begin{Function}{max}
\index{arithmetic}
\begin{verbatim}
MAX([U:number]):number                     noeval, nospread, or macro
\end{verbatim}
   Returns  the largest of the values in U. If two or more values
   are the same the first is returned.
\begin{verbatim}
   MACRO PROCEDURE MAX(U);
     EXPAND(CDR U, 'MAX2);
\end{verbatim}
\end{Function}
%- - - - - - - - - - - - - - - - - - - - - - - - - - - - - - - - - - -
\begin{Function}{max2}
\index{arithmetic}
\begin{verbatim}
MAX2(U:number, V:number):number           eval, spread
\end{verbatim}
   Returns  the larger of U and V. If  U and V are the same value
   U is returned (U and V might be of different types).
\begin{verbatim}

   EXPR PROCEDURE MAX2(U, V);
     IF LESSP(U, V) THEN V ELSE U;
\end{verbatim}
\end{Function}
%- - - - - - - - - - - - - - - - - - - - - - - - - - - - - - - - - - -
\begin{Function}{min}
\index{arithmetic}
\begin{verbatim}
MIN([U:number]):number                     noeval, nospread, or macro
\end{verbatim}
   Returns  the  smallest of  the  values in  U. If  two  or more
   values are the same the first of these is returned.
\begin{verbatim}
   MACRO PROCEDURE MIN(U);
     EXPAND(CDR U, 'MIN2);
\end{verbatim}

\end{Function}
%- - - - - - - - - - - - - - - - - - - - - - - - - - - - - - - - - - -
\begin{Function}{min2}
\index{arithmetic}
\begin{verbatim}
MIN2(U:number, V:number):number           eval, spread
\end{verbatim}
   Returns  the smaller  of its arguments.   If  U and  V are the
   same  value,  U is  returned (U  and V  might be  of different
   types).
\begin{verbatim}

   EXPR PROCEDURE MIN2(U, V);
     IF GREATERP(U, V) THEN V ELSE U;
\end{verbatim}

\end{Function}
%- - - - - - - - - - - - - - - - - - - - - - - - - - - - - - - - - - -
\begin{Function}{minus}
\index{arithmetic}
\begin{verbatim}
MINUS(U:number):number                    eval, spread
\end{verbatim}
   Returns -U.
\begin{verbatim}

   EXPR PROCEDURE MINUS(U);
     DIFFERENCE(0, U);
\end{verbatim}
\end{Function}
%- - - - - - - - - - - - - - - - - - - - - - - - - - - - - - - - - - -
\begin{Function}{plus}
\index{arithmetic}
\begin{verbatim}
PLUS([U:number]):number                    noeval, nospread, or macro
\end{verbatim}
   Forms the sum of all its arguments.
\begin{verbatim}
   MACRO PROCEDURE PLUS(U);
     EXPAND(CDR U, 'PLUS2);
\end{verbatim}
\end{Function}
%- - - - - - - - - - - - - - - - - - - - - - - - - - - - - - - - - - -
\begin{Function}{plus2}
\index{arithmetic}
\begin{verbatim}
PLUS2(U:number, V:number):number          eval, spread
\end{verbatim}
   Returns the sum of U and V.

\end{Function}
%- - - - - - - - - - - - - - - - - - - - - - - - - - - - - - - - - - -
\begin{Function}{quotient}
\index{arithmetic}
\begin{verbatim}
QUOTIENT(U:number, V:number):number       eval, spread
\end{verbatim}
   The  quotient of  U divided  by V  is returned.    Division of
   two  positive or two negative integers  is conventional.  When
   both  U and V are integers and exactly one of them is negative
   the  value returned is the negative truncation of the absolute
   value  of  U divided  by  the absolute  value of  V.  An error
   occurs if division by zero is attempted:
\begin{verbatim}
   ***** Attempt to divide by 0 in QUOTIENT
\end{verbatim}


\end{Function}
%- - - - - - - - - - - - - - - - - - - - - - - - - - - - - - - - - - -
\begin{Function}{remainder}
\index{arithmetic}
\begin{verbatim}
REMAINDER(U:number, V:number):number      eval, spread
\end{verbatim}
   If  both  U  and V  are  integers the  result  is  the integer
   remainder  of U divided by V.  If either parameter is floating
   point,  the  result is  the difference  between U  and V*(U/V)
   all  in  floating point.    If either  number is  negative the
   remainder  is  negative.   If  both are  positive or  both are
   negative  the remainder is positive.   An error occurs if V is
   zero:
\begin{verbatim}
   ***** Attempt to divide by 0 in REMAINDER
\end{verbatim}
\begin{verbatim}

   EXPR PROCEDURE REMAINDER(U, V);
     DIFFERENCE(U, TIMES2(QUOTIENT(U, V), V));
\end{verbatim}
\end{Function}
%- - - - - - - - - - - - - - - - - - - - - - - - - - - - - - - - - - -
\begin{Function}{sub1}
\index{arithmetic}
\begin{verbatim}
SUB1(U:number):number                     eval, spread
\end{verbatim}
   Returns  the value of U less 1.   If U is a FLOAT type number,
   the value returned is U less 1.0.
\begin{verbatim}

   EXPR PROCEDURE SUB1(U);
     DIFFERENCE(U, 1);
\end{verbatim}
\end{Function}
%- - - - - - - - - - - - - - - - - - - - - - - - - - - - - - - - - - -
\begin{Function}{times}
\index{arithmetic}
\begin{verbatim}
TIMES([U:number]):number                   noeval, nospread, or macro
\end{verbatim}
   Returns the product of all its arguments.
\begin{verbatim}
   MACRO PROCEDURE TIMES(U);
     EXPAND(CDR U, 'TIMES2);
\end{verbatim}
\end{Function}
%- - - - - - - - - - - - - - - - - - - - - - - - - - - - - - - - - - -
\begin{Function}{times2}
\index{arithmetic}
\begin{verbatim}
TIMES2(U:number, V:number):number         eval, spread
\end{verbatim}
   Returns the product of U and V.
\end{Function}
%- - - - - - - - - - - - - - - - - - - - - - - - - - - - - - - - - - -



\section{MAP Composite Functions}


%- - - - - - - - - - - - - - - - - - - - - - - - - - - - - - - - - - -
\begin{Function}{map}
\begin{verbatim}
MAP(X:list, FN:function):any              eval, spread
\end{verbatim}
   Applies FN to successive CDR segments of X. NIL is returned.
\begin{verbatim}

   EXPR PROCEDURE MAP(X, FN);
     WHILE X DO << FN X; X := CDR X >>;
\end{verbatim}
\end{Function}
%- - - - - - - - - - - - - - - - - - - - - - - - - - - - - - - - - - -
\begin{Function}{mapc}
\begin{verbatim}
MAPC(X:list, FN:function):any             eval, spread
\end{verbatim}
   FN  is applied  to successive CAR  segments of list  X. NIL is
   returned.
\begin{verbatim}
   EXPR PROCEDURE MAPC(X, FN);
     WHILE X DO << FN CAR X; X := CDR X >>;
\end{verbatim}
\end{Function}
%- - - - - - - - - - - - - - - - - - - - - - - - - - - - - - - - - - -
\begin{Function}{mapcan}
\begin{verbatim}
MAPCAN(X:list, FN:function):any           eval, spread
\end{verbatim}
   A  concatenated list of FN  applied to successive CAR elements
   of X is returned.
\begin{verbatim}
   EXPR PROCEDURE MAPCAN(X, FN);
     IF NULL X THEN NIL
       ELSE NCONC(FN CAR X, MAPCAN(CDR X, FN));
\end{verbatim}
\end{Function}
%- - - - - - - - - - - - - - - - - - - - - - - - - - - - - - - - - - -
\begin{Function}{mapcar}
\begin{verbatim}
MAPCAR(X:list, FN:function):any           eval, spread
\end{verbatim}
   Returned  is a constructed  list of FN applied  to each CAR of
   list X.
\begin{verbatim}
   EXPR PROCEDURE MAPCAR(X, FN);
     IF NULL X THEN NIL
       ELSE FN CAR X . MAPCAR(CDR X, FN);
\end{verbatim}

\end{Function}
%- - - - - - - - - - - - - - - - - - - - - - - - - - - - - - - - - - -
\begin{Function}{mapcon}
\begin{verbatim}
MAPCON(X:list, FN:function):any           eval, spread
\end{verbatim}
   Returned  is a concatenated  list of FN  applied to successive
   CDR segments of X.
\begin{verbatim}
   EXPR PROCEDURE MAPCON(X, FN);
     IF NULL X THEN NIL
       ELSE NCONC(FN X, MAPCON(CDR X, FN));
\end{verbatim}

\end{Function}
%- - - - - - - - - - - - - - - - - - - - - - - - - - - - - - - - - - -
\begin{Function}{maplist}
\begin{verbatim}
MAPLIST(X:list, FN:function):any          eval, spread
\end{verbatim}
   Returns  a constructed  list of  FN applied  to successive CDR
   segments of X.
\begin{verbatim}
   EXPR PROCEDURE MAPLIST(X, FN);
     IFNULL X THEN NIL
       ELSE FN X . MAPLIST(CDR X, FN);
\end{verbatim}
\end{Function}
%- - - - - - - - - - - - - - - - - - - - - - - - - - - - - - - - - - -

\section{Composite Functions}


%- - - - - - - - - - - - - - - - - - - - - - - - - - - - - - - - - - -
\begin{Function}{apend}
\begin{verbatim}
APPEND(U:list, V:list):list               eval, spread
\end{verbatim}
   Returns  a constructed list in which  the last element of U is
   followed  by the first element  of V. The list  U is copied, V
   is not.
\begin{verbatim}
   EXPR PROCEDURE APPEND(U, V);
     IF NULL U THEN V
       ELSE CAR U . APPEND(CDR U, V);
\end{verbatim}
\end{Function}
%- - - - - - - - - - - - - - - - - - - - - - - - - - - - - - - - - - -
\begin{Function}{assoc}
\begin{verbatim}
ASSOC(U:any, V:alist):{dotted-pair, NIL}  eval, spread
\end{verbatim}
   If  U occurs as the CAR portion of  an element of the \nameref{alist} V,
   the  dotted-pair in which U occurred  is returned, else NIL is
   returned.   ASSOC might not detect a poorly formed alist so an
   invalid construction may be detected by CAR or CDR.
\begin{verbatim}
   EXPR PROCEDURE ASSOC(U, V);
     IF NULL V THEN NIL
        ELSE IF ATOM CAR V THEN
             ERROR(000, LIST(V, "is a poorly formed alist"))
        ELSE IF U = CAAR V THEN CAR V
        ELSE ASSOC(U, CDR V);
\end{verbatim}
\end{Function}
%- - - - - - - - - - - - - - - - - - - - - - - - - - - - - - - - - - -
\begin{Function}{deflist}
\begin{verbatim}
DEFLIST(U:dlist, IND:id):list             eval, spread
\end{verbatim}
   A  "dlist" is a  list in which  each element is  a two element
   list:   (ID:id  PROP:any).   Each  ID in  U has  the indicator
   IND  with property  PROP placed  on its  property list  by the
   PUT  function.   The value of  DEFLIST is a list  of the first
   elements  of each two element list.  Like \nameref{put}, DEFLIST may not
   be used to define functions.
\begin{verbatim}
   EXPR PROCEDURE DEFLIST(U, IND);
     IF NULL U THEN NIL
      ELSE << PUT(CAAR U, IND, CADAR U);
              CAAR U >> .  DEFLIST(CDR U, IND);
\end{verbatim}
\end{Function}
%- - - - - - - - - - - - - - - - - - - - - - - - - - - - - - - - - - -
\begin{Function}{delete}
\begin{verbatim}
DELETE(U:any, V:list):list                eval, spread
\end{verbatim}  
   Returns  V with  the first top  level occurrence  of U removed
   from it.
\begin{verbatim}
   EXPR PROCEDURE DELETE(U, V);
     IF NULL V THEN NIL
      ELSE IF CAR V = U THEN CDR V
      ELSE CAR V . DELETE(U, CDR V);
\end{verbatim}
\end{Function}
%- - - - - - - - - - - - - - - - - - - - - - - - - - - - - - - - - - -
\begin{Function}{digit}
\begin{verbatim}
DIGIT(U:any):boolean                      eval, spread
\end{verbatim}
   Returns T if U is a digit, otherwise NIL.
\begin{verbatim}
   EXPR PROCEDURE DIGIT(U);
     IF MEMQ(U, '(!0 !1 !2 !3 !4 !5 !6 !7 !8 !9))
      THEN T ELSE NIL;
\end{verbatim}
\end{Function}
%- - - - - - - - - - - - - - - - - - - - - - - - - - - - - - - - - - -
\begin{Function}{length}
\begin{verbatim}
LENGTH(X:any):integer                     eval, spread
\end{verbatim}
   The top level length of the list X is returned.
\begin{verbatim}
   EXPR PROCEDURE LENGTH(X);
     IF ATOM X THEN 0
      ELSE PLUS(1, LENGTH CDR X);
\end{verbatim}

\end{Function}
%- - - - - - - - - - - - - - - - - - - - - - - - - - - - - - - - - - -
\begin{Function}{liter}
\begin{verbatim}
LITER(U:any):boolean                      eval, spread
\end{verbatim}
   Returns   T  if  U  is  a  character   of  the  alphabet,  NIL
   otherwise.
\begin{verbatim}
   EXPR PROCEDURE LITER(U);
     IF MEMQ(U, '(!A !B !C !D !E !F !G !H !I !J !K !L !M
                  !N !O !P !Q !R !S !T !U !V !W !X !Y !Z
                  !a !b !c !d !e !f !g !h !i !j !k !l !m
                  !n !o !p !q !r !s !t !u !v !w !x !y !z))
\end{verbatim}
     The  published report  omits escape  characters.   These are
   required  for  both  upper  and  lower  case  as  some systems
   default to lower.

\end{Function}
%- - - - - - - - - - - - - - - - - - - - - - - - - - - - - - - - - - -
\begin{Function}{member}
\begin{verbatim}
MEMBER(A:any, B:list):extra-boolean       eval, spread
\end{verbatim}
   Returns  NIL  if A  is not  a member  of  list B,  returns the
   remainder of B whose first element is A.
\begin{verbatim}
   EXPR PROCEDURE MEMBER(A, B);
     IF NULL B THEN NIL
      ELSE IF A = CAR B THEN B
      ELSE MEMBER(A, CDR B);
\end{verbatim}

\end{Function}
%- - - - - - - - - - - - - - - - - - - - - - - - - - - - - - - - - - -
\begin{Function}{memq}
\begin{verbatim}
MEMQ(A:any, B:list):extra-boolean         eval, spread
\end{verbatim}
   Same as \nameref{member} but an \nameref{eq} check is used for comparison.
\begin{verbatim}
   EXPR PROCEDURE MEMQ(A, B);
     IF NULL B THEN NIL
        ELSE IF A EQ CAR B THEN B
        ELSE MEMQ(A, CDR B);
\end{verbatim}

\end{Function}
%- - - - - - - - - - - - - - - - - - - - - - - - - - - - - - - - - - -
\begin{Function}{nconc}
\begin{verbatim}
NCONC(U:list, V:list):list                eval, spread
\end{verbatim}
   Concatenates  V to U without  copying U. The last  CDR of U is
   modified to point to V.
\begin{verbatim}
   EXPR PROCEDURE NCONC(U, V);
   BEGIN SCALAR W;
      IF NULL U THEN RETURN V;
      W := U;
      WHILE CDR W DO W := CDR W;
      RPLACD(W, V);
      RETURN U
   END;
\end{verbatim}
\end{Function}
%- - - - - - - - - - - - - - - - - - - - - - - - - - - - - - - - - - -
\begin{Function}{pair}
\begin{verbatim}
PAIR(U:list, V:list):alist                eval, spread
\end{verbatim}
   U  and  V are  lists which  must have  an identical  number of
   elements.   If not, an error occurs (the 000 used in the \nameref{error}
   call  is arbitrary and need not be adhered to).  Returned is a
   list  where each element is a dotted-pair, the CAR of the pair
   being from U, and the CDR the corresponding element from V.
\begin{verbatim}
   EXPR PROCEDURE PAIR(U, V);
     IF AND(U, V) THEN (CAR U . CAR V) . PAIR(CDR U, CDR V)
      ELSE IF OR(U, V) THEN ERROR(000,
        "Different length lists in PAIR")
      ELSE NIL;
\end{verbatim}
\end{Function}
%- - - - - - - - - - - - - - - - - - - - - - - - - - - - - - - - - - -
\begin{Function}{reverse}
\begin{verbatim}
REVERSE(U:list):list                      eval, spread
\end{verbatim}
   Returns a copy of the top level of U in reverse order.
\begin{verbatim}
   EXPR PROCEDURE REVERSE(U);
   BEGIN SCALAR W;
      WHILE U DO << W := CAR U . W;
                    U := CDR U >>;
      RETURN W
   END;
\end{verbatim}

\end{Function}
%- - - - - - - - - - - - - - - - - - - - - - - - - - - - - - - - - - -
\begin{Function}{sassoc}
\begin{verbatim}
SASSOC(U:any, V:alist, FN:function):any   eval, spread
\end{verbatim}
   Searches  the \nameref{alist} V for  an occurrence of U. If  U is not in
   the alist the evaluation of function FN is returned.
\begin{verbatim}
   EXPR PROCEDURE SASSOC(U, V, FN);
     IF NULL V THEN FN()
      ELSE IF U = CAAR V THEN CAR V
      ELSE SASSOC(U, CDR V, FN);
\end{verbatim}
\end{Function}
%- - - - - - - - - - - - - - - - - - - - - - - - - - - - - - - - - - -
\begin{Function}{sublis}
\index{substitution}
\begin{verbatim}
SUBLIS(X:alist, Y:any):any                eval, spread
\end{verbatim}
  The  value returned is  the result of  substituting the CDR of
   each  element of the  alist X for every  occurrence of the CAR
   part of that element in Y.
\begin{verbatim}
   EXPR PROCEDURE SUBLIS(X, Y);
    IF NULL X THEN Y
      ELSE BEGIN SCALAR U;
                 U := ASSOC(Y, X);
                 RETURN IF U THEN CDR U
                        ELSE IF ATOM Y THEN Y
                        ELSE SUBLIS(X, CAR Y) .
                             SUBLIS(X, CDR Y)
                 END;
\end{verbatim}
\end{Function}
%- - - - - - - - - - - - - - - - - - - - - - - - - - - - - - - - - - -
\begin{Function}{subst}
\index{substitution}
\begin{verbatim}
SUBST(U:any, V:any, W:any):any            eval, spread
\end{verbatim}
   The  value returned  is the result  of substituting  U for all
   occurrences of V in W.
\begin{verbatim}
   EXPR PROCEDURE SUBST(U, V, W);
     IF NULL W THEN NIL
      ELSE IF V = W THEN U
      ELSE IF ATOM W THEN W
      ELSE SUBST(U, V, CAR W) . SUBST(U, V, CDR W);
\end{verbatim}
\end{Function}
%- - - - - - - - - - - - - - - - - - - - - - - - - - - - - - - - - - -

\section{Interpreter}



%- - - - - - - - - - - - - - - - - - - - - - - - - - - - - - - - - - -
\begin{Function}{apply}
\begin{verbatim}
APPLY(FN:{id,function}, ARGS:any-list):any      eval, spread
\end{verbatim}
   APPLY   returns  the  value  of   FN  with  actual  parameters
   ARGS.  The  actual  parameters  in  ARGS  are  already  in the
   form  required  for binding  to the  formal parameters  of FN.
   Implementation  specific  portions  described  in  English are
   enclosed in boxes.
\begin{verbatim}
   EXPR PROCEDURE APPLY(FN, ARGS);
   BEGIN SCALAR DEFN;
     IF---------------------------------------------
       -Spread  the   actual   parameters  in   ARGS-
       -following  the  conventions:    for  calling-
       -functions, transfer  to the  entry  point of;
       -                                            -
       -the function, and return  the value returned-
       ----------------------------------------------
     IF IDP FN THEN RETURN
        IF NULL(DEFN := GETD FN) THEN
           ERROR(000, LIST(FN, "is an undefined function"))
        ELSE IF CAR DEFN EQ 'EXPR THEN
           APPLY(CDR DEFN, ARGS)

        ELSE ERROR(000,
           LIST(FN, "cannot be evaluated by APPLY"));
     IF OR(ATOM FN, NOT(CAR FN EQ 'LAMBDA)) THEN
        ERROR(000,
        LIST(FN, "cannot be evaluated by APPLY"));
     RETURN
        -Bind-the--actual-parameters--in-ARGS--to-the-
        -                                            -
        -formal parameters of the  lambda expression.-
        -If the  two lists  are not  of  equal length-
        -then ERROR(000, "Number of parameters do not-
        -match"); The  value returned  is  EVAL CADDR-
        ----------------------------------------------
   END;
\end{verbatim}
\end{Function}
%- - - - - - - - - - - - - - - - - - - - - - - - - - - - - - - - - - -
\begin{Function}{eval}
\begin{verbatim}
EVAL(U:any):any                           eval, spread
\end{verbatim}
   The  value of  the expression  U is  computed.   Error numbers
   are  arbitrary.   Portions of EVAL  involving machine specific
   coding are expressed in English enclosed in boxes.
\begin{verbatim}
   EXPR PROCEDURE EVAL(U);
   BEGIN SCALAR FN;
      IF CONSTANTP U THEN RETURN U;
      IF IDP U THEN RETURN
        -U-is-an-id.--Return-the-value-most-currently-
        -bound to U or  if there is  no such binding:-
        -                                            -
        ----------------------------------------------
      IF PAIRP CAR U THEN RETURN
         IF CAAR U EQ 'LAMBDA THEN APPLY(CAR U, EVLIS CDR U)
         ELSE ERROR(000, LIST(CAR U,
                    "improperly formed LAMBDA expression"))
         ELSE IF CODEP CAR U THEN
                    RETURN APPLY(CAR U, EVLIS CDR U);

      FN := GETD CAR U;
      IF NULL FN THEN
         ERROR(000, LIST(CAR U, "is an undefined function"))
      ELSE IF CAR FN EQ 'EXPR THEN
         RETURN APPLY(CDR FN, EVLIS CDR U)
      ELSE IF CAR FN EQ 'FEXPR THEN
         RETURN APPLY(CDR FN, LIST CDR U)
      ELSE IF CAR FN EQ 'MACRO THEN
         RETURN EVAL APPLY(CDR FN, LIST U)
   END;
\end{verbatim}
see also \nameref{constantp}, \nameref{idp}, \nameref{pairp},
\nameref{evlis}, nameref{apply}, nameref{getd}
\end{Function}
%- - - - - - - - - - - - - - - - - - - - - - - - - - - - - - - - - - -
\begin{Function}{evlis}
\begin{verbatim}
EVLIS(U:any-list):any-list                eval, spread
\end{verbatim}
   EVLIS returns a list of the evaluation of each element of U.
\begin{verbatim}
   EXPR PROCEDURE EVLIS(U);
     IF NULL U THEN NIL
      ELSE EVAL CAR U . EVLIS CDR U;
\end{verbatim}
see also \nameref{eval}
\end{Function}
%- - - - - - - - - - - - - - - - - - - - - - - - - - - - - - - - - - -
\begin{Function}{expand}
\index{macro}
\begin{verbatim}
EXPAND(L:list, FN:function):list          eval, spread
\end{verbatim}
   FN  is a defined function  of two arguments to  be used in the
   expansion of a \name{macro} defined by \nameref{dm}. EXPAND returns a list in the form:
\begin{verbatim}
   (FN L   (FN L  ...(FN L    L ) ... ))
        0       1         n-1  n
\end{verbatim}
   where  n is the number of elements in L, Li  is the ith element
   of L.
\begin{verbatim}
   EXPR PROCEDURE EXPAND(L,FN);
     IF NULL CDR L THEN CAR L
      ELSE LIST(FN, CAR L, EXPAND(CDR L, FN));
\end{verbatim}
\end{Function}
%- - - - - - - - - - - - - - - - - - - - - - - - - - - - - - - - - - -
\begin{Function}{function}
\begin{verbatim}
FUNCTION(FN:function):function            noeval, nospread
\end{verbatim}
   The  function FN  is to  be passed  to another  function.   If
   FN  is to have  side effects its free  variables must be \nameref{fluid}
   or  \nameref{global}.   FUNCTION is like \nameref{quote} but its  argument may be
   affected  by compilation.  We  do not consider \nameindex{FUNARG}s in this
   report.


\end{Function}
%- - - - - - - - - - - - - - - - - - - - - - - - - - - - - - - - - - -
\begin{Function}{quote}
\begin{verbatim}
QUOTE(U:any):any                          noeval, nospread
\end{verbatim}
   Stops evaluation and returns U unevaluated.
\begin{verbatim}
   FEXPR PROCEDURE QUOTE(U);
     CAR U;
\end{verbatim}
\end{Function}
%- - - - - - - - - - - - - - - - - - - - - - - - - - - - - - - - - - -


\section{Input and Output}
\begin{Introduction}{IO}
The user normally communicates with Standard LISP through
\nameindex{standard devices}. The default devices are selected in accordance
with the conventions of the implementation site. Other input and
output devices or files may be selected for reading and writing using
the functions described herein.
\end{Introduction}


%- - - - - - - - - - - - - - - - - - - - - - - - - - - - - - - - - - -
\begin{Function}{close}
\begin{verbatim}
CLOSE(FILEHANDLE:any):any                 eval, spread
\end{verbatim}
   Closes  the  file with  the  internal name  FILEHANDLE writing
   any  necessary  end of  file marks  and such.    The  value of
   FILEHANDLE  is that  returned by  the corresponding  OPEN. The
   value  returned is the value of FILEHANDLE. An error occurs if
   the file can not be closed.
\begin{verbatim}
    ***** FILEHANDLE could not be closed
\end{verbatim}

\end{Function}
%- - - - - - - - - - - - - - - - - - - - - - - - - - - - - - - - - - -
\begin{Function}{eject}
\begin{verbatim}
EJECT():NIL                               eval, spread
\end{verbatim}
   Skip  to the top  of the next  output page.   Automatic EJECTs
   are  executed by  the print functions  when the  length set by
   the \nameref{pagelength} function is exceeded.

\end{Function}
%- - - - - - - - - - - - - - - - - - - - - - - - - - - - - - - - - - -
\begin{Function}{linelength}
\begin{verbatim}
LINELENGTH(LEN:{integer, NIL}):integer    eval, spread
\end{verbatim}
   If  LEN is  an integer the  maximum line length  to be printed
   before  the print  functions initiate  an automatic nameref{terpri} is
   set  to the value LEN. No initial Standard LISP line length is
   assumed.    The previous line  length is  returned except when
   LEN  is NIL. This special case returns the current line length
   and  does not cause  it to be reset.   An  error occurs if the
   requested  line length is too large for the currently selected
   output file or LEN is negative or zero.
\begin{verbatim}
    ***** LEN is an invalid line length
\end{verbatim}


\end{Function}
%- - - - - - - - - - - - - - - - - - - - - - - - - - - - - - - - - - -
\begin{Function}{lposn}
\begin{verbatim}
LPOSN():integer                           eval, spread
\end{verbatim}
   Returns  the number of lines printed on  the current page.  At
   the top of a page, 0 is returned.

\end{Function}
%- - - - - - - - - - - - - - - - - - - - - - - - - - - - - - - - - - -
\begin{Function}{open}
\index{input}\index{output}
\begin{verbatim}
OPEN(FILE:any, HOW:id):any                eval, spread
\end{verbatim}
   Open  the file with the system  dependent name FILE for output
   if  HOW is \nameref{eq}  to OUTPUT, or input  if HOW is  \name{eq} to INPUT. If
   the  file is opened successfully,  a value which is internally
   associated  with the  file is  returned.   This value  must be
   saved  for  use by  \nameref{wrs} and \nameref{rds}. An  error  occurs if  HOW is
   something  other than  INPUT or  OUTPUT or  the file  can't be
   opened.
\begin{verbatim}
   ***** HOW is not option for OPEN
    ***** FILE could not be opened
\end{verbatim}
Use the \nameref{close} function to close a file.

\end{Function}
%- - - - - - - - - - - - - - - - - - - - - - - - - - - - - - - - - - -
\begin{Function}{pagelength}
\begin{verbatim}
PAGELENGTH(LEN:{integer, NIL}):integer    eval, spread
\end{verbatim}
   Sets  the  vertical  length  (in  lines)  of  an  output page.
   Automatic  page  \nameref{eject}s are  executed  by the  print functions
   when  this length  is reached.    The initial  vertical length
   is  implementation  specific.    The previous  page  length is
   returned.     If  LEN is  0,  no  automatic  page  ejects will
   occur.

\end{Function}
%- - - - - - - - - - - - - - - - - - - - - - - - - - - - - - - - - - -
\begin{Function}{posn}
\begin{verbatim}
POSN():integer                            eval, spread
\end{verbatim}
   Returns  the number of characters in  the output buffer.  When
   the buffer is empty, 0 is returned.

\end{Function}
%- - - - - - - - - - - - - - - - - - - - - - - - - - - - - - - - - - -
\begin{Function}{princ}
\begin{verbatim}
PRINC(U:id):id                            eval, spread
\end{verbatim}
   U  must be a  single character id such  as produced by \nameref{explode}
   or  read by  \nameref{readch} or the  value of \nameref{$eol$}.    The effect is
   the  character U displayed upon  the currently selected output
   device.     The value  of  \name{!$EOL!$} causes  termination  of the
   current line like a call to \nameref{terpri}.

\end{Function}
%- - - - - - - - - - - - - - - - - - - - - - - - - - - - - - - - - - -
\begin{Function}{print}
\begin{verbatim}
PRINT(U:any):any                          eval, spread
\end{verbatim}
   Displays  U in \nameref{read}  readable format and  terminates the print
   line.  The value of U is returned.
\begin{verbatim}
   EXPR PROCEDURE PRINT(U);
      << PRIN1 U; TERPRI(); U >>;
\end{verbatim}
\end{Function}
%- - - - - - - - - - - - - - - - - - - - - - - - - - - - - - - - - - -
\begin{Function}{prin1}
\begin{verbatim}
PRIN1(U:any):any                          eval, spread
\end{verbatim}
   U  is  displayed  in  a  \nameref{read} readable  form.     The  format
   of  display  is  the  result  of  \nameref{explode}  expansion;  special
   characters  are  prefixed  with the  escape  character !,  and
   strings  are  enclosed  in "... ".    Lists  are  displayed in
   list-notation and vectors in vector-notation.

\end{Function}
%- - - - - - - - - - - - - - - - - - - - - - - - - - - - - - - - - - -
\begin{Function}{prin2}
\begin{verbatim}
PRIN2(U:any):any                          eval, spread
\end{verbatim}
   U  is  displayed  upon  the  currently  selected  print device
   but  output  is  not  \nameref{read} readable.     The  value  of U  is
   returned.    Items are displayed  as described  in the \nameref{explode}
   function  with the  exceptions that the  escape character does
   not  prefix special characters and strings are not enclosed in
   "... ".   Lists are displayed in  list-notation and vectors in
   vector-notation.  The value of U is returned.

\end{Function}
%- - - - - - - - - - - - - - - - - - - - - - - - - - - - - - - - - - -
\begin{Function}{rds}
\begin{verbatim}
RDS(FILEHANDLE:any):any                   eval, spread
\end{verbatim}
   Input  from  the currently  selected  input file  is suspended
   and  further input comes  from the file named.   FILEHANDLE is
   a  system dependent  internal name  which is  a value returned
   by  \nameref{open}. If  FILEHANDLE is NIL  the standard  input device is
   selected.    When  end of  file is  reached on  a non-standard
   input  device, the standard input device  is reselected.  When
   end  of file occurs on the  standard input device the Standard
   LISP  reader terminates.  RDS returns the internal name of the
   previously selected input file.
\begin{verbatim}
   ***** FILEHANDLE could not be selected for input
\end{verbatim}
The function name RDS goes back to  "read select";
\end{Function}
%- - - - - - - - - - - - - - - - - - - - - - - - - - - - - - - - - - -
\begin{Function}{read}
\begin{verbatim}
READ():any
\end{verbatim}
   The  next  expression  from the  file  currently  selected for
   input.     Valid  input  forms  are:    vector-notation,  dot-
   notation,  list-notation, numbers, function-pointers, strings,
   and  identifiers  with  escape characters.     Identifiers are
   interned   on  the   \name{oblist}  (see  \nameref{intern})
   READ  returns the
   value  of \nameref{\$eof\$} when the end of the currently selected input
   file is reached.


\end{Function}
%- - - - - - - - - - - - - - - - - - - - - - - - - - - - - - - - - - -
\begin{Function}{readch}
\begin{verbatim}
READCH():id
\end{verbatim}
   Returns  the next  interned character from  the file currently
   selected  for input.   Two  special cases occur.    If all the
   characters  in an  input record have  been read,  the value of
   \nameref{\$eol\$}  is  returned.   If  the file  selected for  input has
   all  been read  the value  of \nameref{\$eof\$}  is returned.   Comments
   delimited by % and end-of-line are not transparent to \nameref{readch}.


\end{Function}
%- - - - - - - - - - - - - - - - - - - - - - - - - - - - - - - - - - -
\begin{Function}{terpri}
\begin{verbatim}
TERPRI():NIL
\end{verbatim}
   The current print line is terminated. The following output
   begins on a new line.

\end{Function}
%- - - - - - - - - - - - - - - - - - - - - - - - - - - - - - - - - - -
\begin{Function}{wrs}
\begin{verbatim}
WRS(FILEHANDLE:any):any                   eval, spread
\end{verbatim}
   Output  to the currently  active output file  is suspended and
   further  output is directed to the  file named.  FILEHANDLE is
   an  internal name  which is returned  by \nameref{open}. The file named
   must  have been opened for  output.  If  FILEHANDLE is NIL the
   standard  output device is selected.  WRS returns the internal
   name of the previously selected output file.
\begin{verbatim}
   ***** FILEHANDLE could not be selected for output
\end{verbatim}
The function name WRS goes back to "write select".
\end{Function}
%- - - - - - - - - - - - - - - - - - - - - - - - - - - - - - - - - - -


\section{LISP Reader}
\begin{Introduction}{LISP Reader}
An EVAL read loop has been chosen to drive a Standard LISP system to
provide a continuity in functional syntax. Choices of messages and the
amount of extra information displayed are decisions left to the
implementor.
\end{Introduction}

\begin{Function}{quit}
\begin{verbatim}
QUIT()
\end{verbatim}
Causes termination of the LISP reader and control to be transferred
to the operating system.
\end{Function}

\section{System GLOBAL Variables}


\begin{Variable}{*comp}
\index{expr}
The value of the global variable !*comp controls whether or not 
\nameref{putd} compiles the
function defined in its arguments before defining it. If !*comp is NIL
the function is defined as an \name{expr}. If !*comp is something else the
function is first compiled. Compilation will produce certain changes
in the semantics of functions particularly \nameref{fluid} type access.
\end{Variable}


\begin{Variable}{emsg*}
Will contain the MESSAGE generated by the last \nameref{error} call.
\end{Variable}


\begin{Variable}{$eof$}
The value of !\$eof!\$ is returned by all input functions when the
end \index{end of file}
of the currently selected input file is reached.
\end{Variable}


\begin{Variable}{$eol$}
The value of !\$eol!\$ is returned by \nameref{readch} when it reaches the end
of \name{readch} \index{end of line}
a logical input record. Likewise \nameref{princ} will terminate its current line
(like a call to \nameref{terpri}) when !\$eol!\$ is its argument.
\end{Variable}

\begin{Variable}{*gc}
\index{garbage collector}
!*gc controls the printing of garbage collector messages.  If NIL no
indication of garbage collection may occur.  If non-NIL various system
dependent messages may be displayed.
\end{Variable}


\begin{Variable}{nil}
\name{nil} is a special global variable. It is protected from being modified
by \nameref{set} or \nameref{setq}. Its value is \name{nil}.
\end{Variable}

\begin{Variable}{*raise}
If \name{!*raise} is non-NIL all characters input through Standard LISP
input/output functions will be raised to upper case. If \name{!*RAISE} is NIL
characters will be input as is.
\end{Variable}


\begin{Variable}{t}
\name{t} is a special global variable. It is protected from being modified
by \nameref{set} or \nameref{setq}. Its value is \name{t}.
\end{Variable}

\end{document}
